\documentclass[12pt,a4paper]{article}\usepackage[]{graphicx}\usepackage[]{color}
%% maxwidth is the original width if it is less than linewidth
%% otherwise use linewidth (to make sure the graphics do not exceed the margin)
\makeatletter
\def\maxwidth{ %
  \ifdim\Gin@nat@width>\linewidth
    \linewidth
  \else
    \Gin@nat@width
  \fi
}
\makeatother

\definecolor{fgcolor}{rgb}{0.345, 0.345, 0.345}
\newcommand{\hlnum}[1]{\textcolor[rgb]{0.686,0.059,0.569}{#1}}%
\newcommand{\hlstr}[1]{\textcolor[rgb]{0.192,0.494,0.8}{#1}}%
\newcommand{\hlcom}[1]{\textcolor[rgb]{0.678,0.584,0.686}{\textit{#1}}}%
\newcommand{\hlopt}[1]{\textcolor[rgb]{0,0,0}{#1}}%
\newcommand{\hlstd}[1]{\textcolor[rgb]{0.345,0.345,0.345}{#1}}%
\newcommand{\hlkwa}[1]{\textcolor[rgb]{0.161,0.373,0.58}{\textbf{#1}}}%
\newcommand{\hlkwb}[1]{\textcolor[rgb]{0.69,0.353,0.396}{#1}}%
\newcommand{\hlkwc}[1]{\textcolor[rgb]{0.333,0.667,0.333}{#1}}%
\newcommand{\hlkwd}[1]{\textcolor[rgb]{0.737,0.353,0.396}{\textbf{#1}}}%
\let\hlipl\hlkwb

\usepackage{framed}
\makeatletter
\newenvironment{kframe}{%
 \def\at@end@of@kframe{}%
 \ifinner\ifhmode%
  \def\at@end@of@kframe{\end{minipage}}%
  \begin{minipage}{\columnwidth}%
 \fi\fi%
 \def\FrameCommand##1{\hskip\@totalleftmargin \hskip-\fboxsep
 \colorbox{shadecolor}{##1}\hskip-\fboxsep
     % There is no \\@totalrightmargin, so:
     \hskip-\linewidth \hskip-\@totalleftmargin \hskip\columnwidth}%
 \MakeFramed {\advance\hsize-\width
   \@totalleftmargin\z@ \linewidth\hsize
   \@setminipage}}%
 {\par\unskip\endMakeFramed%
 \at@end@of@kframe}
\makeatother

\definecolor{shadecolor}{rgb}{.97, .97, .97}
\definecolor{messagecolor}{rgb}{0, 0, 0}
\definecolor{warningcolor}{rgb}{1, 0, 1}
\definecolor{errorcolor}{rgb}{1, 0, 0}
\newenvironment{knitrout}{}{} % an empty environment to be redefined in TeX

\usepackage{alltt}\usepackage[]{graphicx}\usepackage[]{color}
%% maxwidth is the original width if it is less than linewidth
%% otherwise use linewidth (to make sure the graphics do not exceed the margin)
\makeatletter
\def\maxwidth{ %
  \ifdim\Gin@nat@width>\linewidth
    \linewidth
  \else
    \Gin@nat@width
  \fi
}
\makeatother

\usepackage{alltt}%
\usepackage[]{graphicx}\usepackage[]{color}
%% maxwidth is the original width if it is less than linewidth
%% otherwise use linewidth (to make sure the graphics do not exceed the margin)

%2multibyte Version: 5.50.0.2960 CodePage: 1252
\usepackage{amsfonts}
\usepackage{amssymb}
\usepackage[centertags]{amsmath}
\usepackage{graphicx}%
\usepackage{natbib}
\usepackage{color}
\usepackage[dvipsnames,svgnames*]{xcolor}
\usepackage{array}
\usepackage[hidelinks]{hyperref}
\usepackage[font=small,skip=5pt]{caption}
\usepackage[aboveskip=2pt]{subcaption}
\usepackage{amsmath}
\usepackage{amsthm}
%\usepackage{tikz}
%\usetikzlibrary{bayesnet}
\usepackage{url}
\usepackage{ulem}
\usepackage{afterpage}
\setcounter{MaxMatrixCols}{30}

\setlength{\topmargin}{0in}
\setlength{\oddsidemargin}{0.1in}
\setlength{\evensidemargin}{0.1in}
\setlength{\textwidth}{6.5in}
\setlength{\textheight}{8.25in}
\setlength\parindent{0pt}
\IfFileExists{upquote.sty}{\usepackage{upquote}}{}
\IfFileExists{upquote.sty}{\usepackage{upquote}}{}
\begin{document}

\section{Introduction}

\begin{itemize}
\item Self driving cars becoming more common
\item Cars are easy capable of driving by themselves if there is no traffic around them
\item Cars need to be able to predict movement of other cars
\item Drivers display differing behaviour, able to group drivers into different categories
\item Heteogeneity between and within groups
\item Need to deal with new cars as they are observed
\end{itemize}

\section{Data processing}

\begin{itemize}
\item Modern cars are able to track lane marking and can calculate distance to other cars in relative co-ordinates (Thuy and Leon 2010)
\item Data from NGSIM is in coordinates relative to start of road - no impact of the curvature of the lane
\item Fit curve to car data to estimate lane midpoints (Woo et. al. 2016) 
\begin{enumerate}
\item At each point in time $t$ calculate the total distance each car has travelled along the road using $d_{i, t} = \sum_{s=1}^t v_{i, s}$, where $v_{i, s}$ is the velocity of car $i$ at time $s$. 
\item Estimate polynomial functions of the form $x_{i, t} = f(d_{i, t})$ and $y_{i, t} = g(d_{i, t})$, where $x_{i, t}$ and $y_{i, t}$ are the car coordinates relative to the start of the road.
\item Use polynomials to get $\hat{x}_{i, t}$ and $\hat{y}_{i, t}$ as estimates of the midpoint of the lane after travelling a distance $d_{i, t}$.
\item Define new coordinates 
\begin{align}
x^*_{i, t} &= \sqrt{(x_{i, t}-\hat{x}_{i, t})^2 + (y_{i, t} - \hat{y}_{i, t})^2)} \label{xRel} \\
y^*_{i, t} &= d_{i, t} \label{yRel}
\end{align}
\end{enumerate}
\item Note: This should be calculated separately for each lane.
\end{itemize}

\section{Hierarchical Motion Model}

The position of car $i$ at time $t$ can be fully determined by its inital position, $\{x^*_{i, 0}, y^*_{i, 0}\}$ and the history of the driver's inputs: the cars velocity $v_{i, t}$ and steering angle, $\delta_{i, t}$ by
\begin{align}
x^*_{t} &= x^*_{t-1} + v_{t} \cos(\delta_{t}) \label{xEq} \\
y^*_{t} &= y^*_{t-1} + v_{t} \sin(\delta_{t}). \label{yEq},
\end{align}
where $\delta_{t} = \pi/2$ denotes that the car has no lateral movement and $x^*_{t} = x^*_{t-1}$.
From this relationship, the key inputs can be extracted by
\begin{align}
\delta_{t} &= \tan^{-1}\left(\frac{(y^*_{t} - y^*_{t-1})}{(x^*_{t} - x^*_{t-1})} \right) \label{dEq} \\
v_{t} &= \sqrt{(x^*_{t} - x^*_{t-1})^2 + (y^*_{t} - y^*_{t-1})^2} \label{vEq} \\
a_{t} &= v_{t} - v_{t-1}. \label{aEq}
\end{align}
where $a_{t}$ denotes the acceleration at time $t$.

Auto-regressive processes of orders $p$ and $q$ for $a_i$ and  $\delta_i$ are used to model the level of persistance and variation of driver actions, with
\begin{align}
a_{t} &= \sum_{j = 1}^p \phi_{j} a_{t-j} + \epsilon_{t} \label{aAR} \\
\delta_{t} &= \pi/2 + \sum_{j = 1}^q \gamma_{j} (\delta_{t-j} - \pi/2) + \eta_{t} \label{dAR}
\end{align}
assuming that $\epsilon_{t}$ are independently and identically distributed according to a Student's t distribution with variance $\sigma^{2}_{\epsilon}$ and degrees of freedom $\nu_{\epsilon}$, and $\eta_{t}$ similarly has a Student't t distribution with variance $\sigma^{2}_{\eta}$ and degrees of freedom $\nu_{\eta}$. The parameter vector to be estimated is collected as $\theta = \{\phi_{1}, \dots, \phi_{p}, \gamma_{1}, \dots, \gamma_{q}, \log(\sigma^{2}_{\epsilon}), \log(\nu_{\epsilon}), \log(\sigma^{2}_{\eta}), \log(\nu_{\eta}) \} \in \mathbb{R}^{p + q + 4}$.

Multiple groups of differing driver behaviour can be allowed while still retaining heterogeneity with each group by jointly estimating each vehicle driver's individual parameter vector, $\theta_i$, as part of a heirarchical model, independently drawing $\theta_i$ from a $K$ component mixture of multivariate normal distributions, augmenting each $\theta_i$ with a mixture component $k_i = 1, \dots, K$ such that 
\begin{equation}
\theta_i | k_i = j \sim N(\mu_j, \Sigma_j).
\end{equation}
For each $j = 1, \dots, K$, multivariate normal priors are chosen for $\mu_j$ with mean $\bar{\mu}_j$ and variance matrix $\Omega_j$, Inverse Wishart priors are chosen for $\Sigma_j$ with degrees of freedom $\tau_j$ and scale matrix $\Psi_j$, and finally $k_{i, j}$ has a Dirichlet prior with parameters $\alpha_{i, j} = \alpha_i$. This can be estimated with MCMC methods.

\section{Online Classification}

Notation:
\begin{itemize}
\item $y_{i, 1:T} = \{\delta_{i, s}, a_{i, s} | s = 1, 2, \dots, T\}$ 
\item $y_{1:N} = \{y_{i, 1:T} | i = 1, 2, \dots N\}$
\item $q(\theta_i | y_{i, 1:T}) \approx p(\theta_i | y_{i, 1:T}, y_{1:N})$, $\lambda$ parameters and dependence on other cars suppressed - more important to be clear about which observations from the new vehicle are used to construct the approximation.
\item $\beta = \{\mu_i, \Sigma_i, i, = 1, 2, \dots, K)$.
\end{itemize}

Modelling process:
\begin{itemize}
\item Fit heirarchical model from section 3 for cars $i = 1, \dots N$ before car starts driving - eg. During manufacturing process so unlimited time available.
\item Observe a new car, say car $N + 1$.
\item Want to learn about parameters $\theta_{N+1}$ and which group $k_{N+1}$ the driver belongs to
\item $p(\theta_{N+1}, k_{N+1} | y_{1:N+1}) \propto p(y_{N+1, 1:T} | \theta_{N+1}) p(\theta_{N+1} | k_{N+1} \beta)p(k_{N+1}) p(\beta | y_{1:N})$
\item Essentially using mixture distribution posterior fit for cars 1 to N as prior for $\theta_{N+1}$.
\item However we want to be able to do this as we observe $y_{N+1}$, eg: Fit distribution to $p(\theta_{N+1}, k_{N+1} | y_{N+1, 1:10}, y_{1:N})$, then to $p(\theta_{N+1}, k_{N+1} | y_{N+1, 1:11}, y_{1:N})$ etc.
\item To do this in VB we need to make two further approximations:
\begin{enumerate}
\item Fit a parametric distribution to the MCMC posterior
\item Replace each  $p(\theta_{N+1} | y_{i, 1:S}, y_{1:N})$ used to fit $q(\theta_{N+1} | y_{N+1, 1:S+1})$ with the VB fit from the last step, $q(\theta_{N+1} | y_{N+1, 1:S})$.
\end{enumerate}
\end{itemize}

\section{Empirical}

\section{Conclusion}

\end{document}