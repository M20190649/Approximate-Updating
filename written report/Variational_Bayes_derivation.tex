\documentclass[12pt]{article}\usepackage[]{graphicx}\usepackage[]{color}
%% maxwidth is the original width if it is less than linewidth
%% otherwise use linewidth (to make sure the graphics do not exceed the margin)
\makeatletter
\def\maxwidth{ %
  \ifdim\Gin@nat@width>\linewidth
    \linewidth
  \else
    \Gin@nat@width
  \fi
}
\makeatother

\definecolor{fgcolor}{rgb}{0.345, 0.345, 0.345}
\newcommand{\hlnum}[1]{\textcolor[rgb]{0.686,0.059,0.569}{#1}}%
\newcommand{\hlstr}[1]{\textcolor[rgb]{0.192,0.494,0.8}{#1}}%
\newcommand{\hlcom}[1]{\textcolor[rgb]{0.678,0.584,0.686}{\textit{#1}}}%
\newcommand{\hlopt}[1]{\textcolor[rgb]{0,0,0}{#1}}%
\newcommand{\hlstd}[1]{\textcolor[rgb]{0.345,0.345,0.345}{#1}}%
\newcommand{\hlkwa}[1]{\textcolor[rgb]{0.161,0.373,0.58}{\textbf{#1}}}%
\newcommand{\hlkwb}[1]{\textcolor[rgb]{0.69,0.353,0.396}{#1}}%
\newcommand{\hlkwc}[1]{\textcolor[rgb]{0.333,0.667,0.333}{#1}}%
\newcommand{\hlkwd}[1]{\textcolor[rgb]{0.737,0.353,0.396}{\textbf{#1}}}%

\usepackage{framed}
\makeatletter
\newenvironment{kframe}{%
 \def\at@end@of@kframe{}%
 \ifinner\ifhmode%
  \def\at@end@of@kframe{\end{minipage}}%
  \begin{minipage}{\columnwidth}%
 \fi\fi%
 \def\FrameCommand##1{\hskip\@totalleftmargin \hskip-\fboxsep
 \colorbox{shadecolor}{##1}\hskip-\fboxsep
     % There is no \\@totalrightmargin, so:
     \hskip-\linewidth \hskip-\@totalleftmargin \hskip\columnwidth}%
 \MakeFramed {\advance\hsize-\width
   \@totalleftmargin\z@ \linewidth\hsize
   \@setminipage}}%
 {\par\unskip\endMakeFramed%
 \at@end@of@kframe}
\makeatother

\definecolor{shadecolor}{rgb}{.97, .97, .97}
\definecolor{messagecolor}{rgb}{0, 0, 0}
\definecolor{warningcolor}{rgb}{1, 0, 1}
\definecolor{errorcolor}{rgb}{1, 0, 0}
\newenvironment{knitrout}{}{} % an empty environment to be redefined in TeX

\usepackage{alltt}
\usepackage{natbib}
\usepackage{color}
\usepackage[dvipsnames,svgnames*]{xcolor}
\usepackage{array}
\usepackage[colorlinks=TRUE, linkcolor=blue]{hyperref}
\usepackage[font=small,skip=5pt]{caption}
\usepackage[aboveskip=2pt]{subcaption}
\usepackage{graphicx}
\usepackage{amsmath}
\usepackage{amsthm}
\usepackage{amsfonts}
\usepackage{url}
\usepackage{ulem}
\usepackage{afterpage}
\usepackage{bbm}

\graphicspath{{images/}}
\DeclareGraphicsRule{.tif}{png}{.png}{`convert #1 `dirname #1`/`basename #1 .tif`.png}
\IfFileExists{upquote.sty}{\usepackage{upquote}}{}
\begin{document}

\title{Derivation of Variational Bayes}

\section{The Evidence Lower Bound}

We are interested in finding the distribution $q(\theta)$ that best approximates the true posterior $p(\theta|y)$, with the best approximation taken as the $q(\theta)$ that minimises the Kullback-Leibler divergence with the posterior, defined as

$$KL[q(\theta)|p(\theta|y)] = \int q(\theta) \ln \left( \frac{q(\theta)}{p(\theta|y)}\right) d\theta.$$

Beginning with the unknown constant $p(y)$, we have that

\begin{eqnarray}
\ln(p(y)) & = & \int_{\theta} q(\theta) \ln(p(y)) d\theta \nonumber \\
& = & \int_{\theta} q(\theta) \ln  \left( \frac{p(y, \theta)}{p(\theta | y)} \right) d\theta \nonumber \\
& = & \int_{\theta} q(\theta) \ln  \left( \frac{p(y, \theta)}{p(\theta | y)} \frac{q(\theta)}{q(\theta)}\right) d\theta \nonumber \\
& = & \int_{\theta} q(\theta) \ln  \left( \frac{q(\theta)}{p(\theta | y)}\right) + \ln \left( \frac{p(y, \theta)}{q(\theta)} \right) d\theta \nonumber \\
& = & \int_{\theta} q(\theta) \ln  \left( \frac{q(\theta)}{p(\theta | y)}\right) d\theta + \int_{\theta} q(\theta) \ln \left( \frac{p(y, \theta)}{q(\theta)} \right) d\theta \nonumber \\
& = & KL [q(\theta) || p(\theta | y)] + F(q, y) \nonumber
\end{eqnarray}

where 
\begin{eqnarray*}
F(q, y) & = & \int_{\theta} q(\theta) \ln \left( \frac{p(y, \theta)}{q(\theta)} \right) d\theta \\
& = &  \int_{\theta} q(\theta) \ln (p(y, \theta)) d\theta -  \int_{\theta} q(\theta) \ln (q(\theta)) d\theta.
\end{eqnarray*}

As $\ln(p(y))$ is constant, the distribution $q(\theta)$ that minimises the KL divergence can be found by maximising $F(q, y)$, which is called the Evidence Lower Bound (ELBO, or $\mathcal{L}(q, y)$) in the machine learning literature. We are yet to make any assumptions about the distribution of $q(\theta)$.

\section{The Mean Field Assumption}
The mean-field assumption is that $q(\theta)$ can be factorised as $q(\theta) = \prod_{i} q_{i}(\theta_{i})$, where each $i$ may be a scalar or vector.
The mean-field derivations follows, assuming that $q(\theta) = q_{1}(\theta_{1}) q_{2}(\theta_{2})$. 

\begin{eqnarray}
F(q, y) & = &\int_{\theta} q_{1}q_{2} \ln (p(y, \theta)) d\theta - \int_{\theta} q_{1}q_{2} \ln (q_{1}q_{2}) d\theta \nonumber \\
& = & \int_{\theta} q_{1}q_{2} ( \ln (p(y, \theta)) - \ln(q_{1})) d\theta - \int_{\theta} q_{1}q_{2} \ln (q_{2}) d\theta \\
& = & \int_{\theta_{1}} q_{1} \left( \int_{\theta_{2}} q_{2} \ln (p(y, \theta )) d\theta_{2}- \ln(q_{1}) \right) d\theta_{1} - \int_{\theta_{1}} q_{1} \int_{\theta_{2}} q_{2} \ln(q_{2}) d\theta_{2} d\theta_{1} \nonumber \\
& = & \int_{\theta_{1}} q_{1} \ln \left( \frac{\exp(  \mathbb{E}_{q_2} [\ln(p(y,\theta))])}{q_{1}} \right) d\theta_{1} + c \\
& = & -KL (q_{1} || \exp( \mathbb{E}_{q_2} [\ln(p(y,\theta))])) + c \nonumber
\end{eqnarray}

(1) uses the fact that
$$ \int_{\theta_{2}} q_{2} \ln(q_{1}) d\theta_{2} = \ln(q_{1}),$$

and (2) uses the fact that
$$ \int_{\theta_{1}} q_{1} \int_{\theta_{2}} q_{2} \ln(q_{2}) d\theta_{2} d\theta_{1} = \mathbb{E}_{q_1}\left[  \int_{\theta_{2}} q_{2} \ln(q_{2}) d\theta_{2}\right] =  \int_{\theta_{2}} q_{2} \ln(q_{2}) d\theta_{2}, $$

a constant term with respect to $q_{1}$.

Maximisation with respect to $q_{1}$ is simple due to the KL divergence term, as we can minimise 
$$KL (q_{1} || \exp(E_{q_{2}} (\ln(p(y,\theta)))))$$
by setting 
$$q_{1} \propto \exp(E_{q_{2}} (\ln(p(y,\theta))))$$

for every $i$ in $\theta$.

\section{Copula Variational Bayes}

If we attach a copula function to our approximation, so that $q(\theta) = q_1(\theta_1) q_2(\theta_2) c(Q_1(\theta_1), Q_2(\theta_2)$, we get the Evidence Lower Bound of
\begin{eqnarray*}
F(q, y) & = & \int_{\theta} q_{1}q_{2}c(Q_1, Q_2) \ln (p(y, \theta)) d\theta - \int_{\theta} q_{1}q_{2}c(Q_1, Q_2) \ln (q_{1}q_{2}c(Q_1, Q_2)) d\theta \\
& = & \int_{\theta_1} q_{1}c(Q_1, Q_2) \left( \int_{\theta_2} q_{2}\ln (p(y, \theta))d\theta_2 - \ln(q_{1}) \right) d\theta_1 - \int_{\theta} q_{1}q_{2}c(Q_1, Q_2) \ln (q_{2}c(Q_1, Q_2)) d\theta \\
& = & \int_{\theta_{1}} q_{1}c(Q_1, Q_2) \ln \left( \frac{\exp(  \mathbb{E}_{q_2} [\ln(p(y,\theta))])}{q_{1}} \right) d\theta_{1} + f(q_{1}, q_{2}).
\end{eqnarray*}

This derivation runs into two problems: the first term is no longer a KL divergence, and the second term is no longer constant with respect to $q_{1}$.
If, between the first and second lines we moved $q_{1}q_{2}c(Q_1, Q_2) \ln(q_{1}c(Q_1, Q_2))$ from the second term to the first, instead of $q_{1}q_{2}c(Q_1, Q_2) \ln(q_{1})$ we could attempt to the match the denominator with the $ q_{1}c(Q_1, Q_2)$ term and make something in the form of a KL divergence.

We would have had
\begin{eqnarray*}
F(q, y) & = & \int_{\theta_1} q_{1}c(Q_1, Q_2) \left( \int_{\theta_2} q_{2}\ln (p(y, \theta))d\theta_2 - \int_{\theta_{2}} q_{2} \ln(q_{1}c(Q_1, Q_2)) d\theta_{2}\right) d\theta_1 \\
& - & \int_{\theta} q_{1}q_{2}c(Q_1, Q_2) \ln (q_{2}) d\theta
\end{eqnarray*}
and still would not get the KL divergence term as the result used in (1) would no longer hold as
$$ \int_{\theta_{2}} q_{2} \ln(q_{1}c(Q_1, Q_2)) d\theta_{2} \neq \ln(q_{1}c(Q_1, Q_2)).$$

Even if we could do this, the second term would be
\begin{eqnarray*}
\int_{\theta} q_{1}q_{2}c(Q_1, Q_2) \ln (q_{2}) d\theta & = &  \int_{\theta_1} q_{1} \int_{\theta_2} q_{2}c(Q_1, Q_2) \ln (q_{2}) d\theta_2 d\theta_1 \\
& = & \mathbb{E}_{q_1} \left [\int_{\theta_2} q_{2}c(Q_1, Q_2) \ln (q_{2}) d\theta_2 \right]
\end{eqnarray*}
which is not constant with respect to $q_{1}$.

Clearly the mean field approach to the derivation can not be directly adapted.




\end{document}
