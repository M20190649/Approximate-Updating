\documentclass[12pt,a4paper]{article}%
\usepackage[]{graphicx}\usepackage[]{color}
%% maxwidth is the original width if it is less than linewidth
%% otherwise use linewidth (to make sure the graphics do not exceed the margin)

%2multibyte Version: 5.50.0.2960 CodePage: 1252
\usepackage{amsfonts}
\usepackage{amssymb}
\usepackage[centertags]{amsmath}
\usepackage{graphicx}%
\usepackage{natbib}
\usepackage{color}
\usepackage[dvipsnames,svgnames*]{xcolor}
\usepackage{array}
\usepackage[hidelinks]{hyperref}
\usepackage[font=small,skip=5pt]{caption}
\usepackage[aboveskip=2pt]{subcaption}
\usepackage{amsmath}
\usepackage[]{algorithm2e}
\usepackage{amsthm}
\usepackage{url}
\usepackage{wasysym}
\usepackage{ulem}
\usepackage{tikz}
\usetikzlibrary{bayesnet}
\usepackage{afterpage}
\usepackage{bbm}
\setcounter{MaxMatrixCols}{30}
\providecommand{\U}[1]{\protect\rule{.1in}{.1in}}
\newtheorem{theorem}{Theorem}
\newtheorem{acknowledgement}[theorem]{Acknowledgement}
\newtheorem{axiom}[theorem]{Axiom}
\newtheorem{case}[theorem]{Case}
\newtheorem{claim}[theorem]{Claim}
\newtheorem{conclusion}[theorem]{Conclusion}
\newtheorem{condition}[theorem]{Condition}
\newtheorem{conjecture}[theorem]{Conjecture}
\newtheorem{corollary}[theorem]{Corollary}
\newtheorem{criterion}[theorem]{Criterion}
\newtheorem{definition}[theorem]{Definition}
\newtheorem{example}[theorem]{Example}
\newtheorem{exercise}[theorem]{Exercise}
\newtheorem{lemma}[theorem]{Lemma}
\newtheorem{notation}[theorem]{Notation}
\newtheorem{problem}[theorem]{Problem}
\newtheorem{proposition}[theorem]{Proposition}
\newtheorem{remark}[theorem]{Remark}
\newtheorem{solution}[theorem]{Solution}
\newtheorem{summary}[theorem]{Summary}
\setlength{\topmargin}{0in}
\setlength{\oddsidemargin}{0.1in}
\setlength{\evensidemargin}{0.1in}
\setlength{\textwidth}{6.5in}
\setlength{\textheight}{8.25in}
\numberwithin{equation}{section}
\title{Real Time Variational Density Forecasting}
\author{Nathaniel Tomasetti, Catherine Forbes and Anastasios Panagiotelis}
\begin{document}

\begin{figure}[h]
\centering
\tikz{ %
    \node[latent] (theta1) {$\theta_1$} ; %
    \node[latent, right = of theta1] (theta2) {$\theta_2$} ; %
    \node[latent, right = of theta2] (theta3) {$\theta_3$} ; %
    \node[latent, below = of theta2] (theta4) {$\theta_4$} ; %
    \edge [-] {theta1, theta3, theta4}  {theta2} ; %
    %\edge{theta2} {theta3} ; %
    %\edge{theta2} {theta4} ; %
    
    \node[latent, below = of theta4] (theta2theta3) {$\theta_2 \theta_3$} ; %
    \node[latent, left = of theta2theta3] (theta1theta2) {$\theta_1 \theta_2$} ; %
    \node[latent, right = of theta2theta3] (theta2theta4) {$\theta_2 \theta_4$} ; %
    \edge [-] {theta1theta2, theta2theta4} {theta2theta3} ; %
    %\edge{theta2theta3} {theta2theta4} ; %
    
    \node[latent, below = of theta1theta2] (t1t3t2) {$\theta_1 \theta_3 | \theta_2$} ; %
    \node[latent, below = of theta2theta4] (t3t4t2) {$\theta_3 \theta_4 | \theta_2$} ; %
    \edge [-] {t1t3t2} {t3t4t2} ;%
    
    \node[latent, below = of t1t3t2] (all) {$\theta_1, \theta_4 | \theta_2, \theta_3$} ; %
}
\end{figure}


\end{document}